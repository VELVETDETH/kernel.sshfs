\section{Netlink Communication}

Only 2 APIs are important in this section.

\subsection{Send Command and Stall}

\begin{lstlisting}
static int netlink_send_command(cmdtype cmd,
                                char *params,
                                struct iovec *iov)
{
	...
    cmd_msg_buffer.cmd    = 
		kzalloc(sizeof(char) * cmdstr_len, GFP_KERNEL);
    cmd_msg_buffer.cmdlen = cmdstr_len;
    cmd_msg_buffer.exec   = false;
    cmd_msg_buffer.msglen = 0;
    cmd_msg_buffer.msg    = NULL;

    strcpy(cmd_msg_buffer.cmd, cmdstr);

    cmd_msg_buffer.ready  = true;
    while(!cmd_msg_buffer.exec)
        msleep(1000);
	...
    iov->iov_len = cmd_msg_buffer.msglen+1;
    iov->iov_base =
         (char *) kzalloc((iov->iov_len), GFP_KERNEL);
    strcpy(iov->iov_base, cmd_msg_buffer.msg);
    
	...
    kzfree(cmd_msg_buffer.msg);
    cmd_msg_buffer.msg = NULL;

    return 0;
}
\end{lstlisting}

In this function, we'll first fulfill the \lstinline{cmd} buffer in the global parameter:\\ \lstinline{cmd_msg_buffer}, and set the \lstinline{ready} bool to \lstinline{true}. Then this procedure will stall until the \lstinline{exec} is \lstinline{true}.

After the stall is finished, this function will copy back the message from the \lstinline{msg} buffer. How to use that message is specified by different methods. Take the \lstinline{sshfs_file_read} as an instance:
\newpage
\begin{lstlisting}[title=\bfseries file read]
static ssize_t sshfs_file_read(struct file* flip, 
	char __user *buf, size_t len, loff_t *ppos)
{
 	...  
    netlink_send_command(Cmd_read, params, iov);
    
    if(copy_to_user(buf, (void*)iov->iov_base, iov->iov_len))
        return -EFAULT;

    *ppos += iov->iov_len;
    ret = (ssize_t) iov->iov_len - 1;

    kzfree(params);
    kzfree(iov);
    return ret;
}
\end{lstlisting}

This one just copy the received message to the buffer required by the \lstinline{read()} VFS API.

\subsection{Message Receiving}

It's necessary to issue that there're 3 steps in this module:
\begin{enumerate}
\item First of all, this method will be called only if there's a message sent from SSH daemon. The message is contained in \lstinline{struct sk_buff *skb}.
\item This method will parse the message from SSH daemon next, there're 3 kind of status:
	\begin{center}
	\begin{tabular}{|l|l|}
	\hline
	Status	& Meaning\\
	\hline
	READY 		& SSH daemon is working and has nothing to do.\\
	message		& SSH daemon is transferring message and \\
				& the content of received message have not yet finished.\\
	DONE 		& SSH daemon has finished transferring.\\
	\hline
	\end{tabular}
	\end{center}
\item At last, send the response to SSH Daemon by \lstinline{nlmsg_unicase}.
	\begin{center}
	\begin{tabular}{|l|l|}
	\hline
	Status	& Meaning\\
	\hline
	NOTHING 		& If SSH daemon is ready and the file system has no \\
				& workload\\
	COPIED		& If SSH daemon is transferring message and the data \\
				& is copied.\\ 
	OK 			& SSH daemon has finished transferring and response to \\
				& it.\\
	\hline
	\end{tabular}
	\end{center}
\end{enumerate}

The codes are listed below:
\newpage
\begin{lstlisting}[title=\bfseries Message Receiving]
static void netlink_recv_msg(struct sk_buff *skb)
{
    struct nlmsghdr *nlh;
    int pid;
	...   

    nlh = (struct nlmsghdr*) skb->data;
    pid = nlh->nlmsg_pid;
    recvmsg = (char *) nlmsg_data(nlh);

    if (!strcmp(recvmsg, recv_ready_str))
        ... // When SSH Daemon is Ready
    else if (!strcmp(recvmsg, recv_done_str))
    	... // When SSH Daemon is Done
    else
    	... // When SSH Daemon is transferring data.
    skb_out = nlmsg_new(msg_size, 0);
    nlh = nlmsg_put(skb_out, 0, 0, NLMSG_DONE, msg_size, 0);
    NETLINK_CB(skb_out).dst_group = 0;
    strncpy(nlmsg_data(nlh), msg, msg_size);

    res = nlmsg_unicast(nl_sk, skb_out, pid);
}
\end{lstlisting}
